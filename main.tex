\documentclass[handout]{beamer} 

\usepackage{amsmath,amssymb,amsfonts,dcolumn,color,graphicx,graphics,setspace,latexsym,setspace,lscape,subfigure,placeins,epsfig,hyperref}
\usepackage{eulervm}
 

\usetheme{Kalgan}

\setbeamercovered{highly dynamic}
\setbeamersize{text margin left=10pt}

\newcommand{\be}{\begin{equation}}
\newcommand{\ee}{\end{equation}}
\newcommand{\lb}{\left}
\newcommand{\rb}{\right}

\title{Riprogettazione della dashboard\\``Situazione Italia''}
\author{X-Spark}
\institute[UniBo]{Alma Mater Studiorum Università di Bologna}
\date{}


\begin{document}
	\begin{frame}[plain]
	  \titlepage
	\end{frame}
	\begin{frame}
  		\frametitle{Outline}
		\tableofcontents
	\end{frame}
	
	\section{Introduzione}

		\begin{frame}
			\frametitle{Il problema}
			Durante la pandemia Covid-19, la popolazione ha scoperto l'importanza di tenersi aggiornata quotidianamente sull'andamento dei dati.\newline \newline
			Si è cominciato con canali non ufficiali per arrivare, a oggi, a comunicazioni giornaliere da parte dei più importanti giornali e telegiornali.
		\end{frame}

		\begin{frame}
			\frametitle{Giornalismo}
			\`E quindi diventato il giornalista la persona che si occupa di comunicare alla popolazione come la pandemia sta evolvendo giorno dopo giorno.\newline \newline
			\`E quindi questa figura professionale  a dover districarsi tra i moltissimi strumenti per l'analisi dei dati:
			\begin{itemize}[<+->]
				\item Bollettini giornalieri\\
				\item Dashboard\\
				\item Fogli Excel\\
				\item Lanci di agenzia\\
			\end{itemize}
		\end{frame}

		\begin{frame}
			\frametitle{Scarsa qualità del giornalismo}
			\begin{columns}[t]
				\begin{column}[T]{5cm}
					La grande richiesta di articoli ha portato a una loro scarsa qualità
				\end{column}

				\begin{column}[T]{5cm}
					[IMMAGINE ARTICOLO BRUTTO]
				\end{column}
			\end{columns}
		\end{frame}

		\begin{frame}
			\frametitle{Gli strumenti e il modus operandi}
			\framesubtitle{L'importanza e la richiesta di avere ottimi strumenti e il modo di lavorare}
			Abbiamo intervistato diversi giornalisti, sia telefonicamente sia tramite un Google Form raccogliendo quelli che sono i metodi di lavori e i desideri per migliorare e semplificare il loro lavoro.
			%\includegraphics{}
			[IMMAGINE DI ALCUNI RISULTATI?]
		\end{frame}

		\begin{frame}
			\frametitle{Task frequenti e significativi}
			\begin{itemize}[<+->]
				\item Comprendere l'andamento della curva epidemiologica\\
				\item Monitorare l'occupazione delle strutture sanitarie\\
				\item Monitorare l'andamento del tasso di letalità\\
				\item Monitorare come i numeri della pandemia si distribuiscono sulla popolazione\\
				\item Confrontare l'andamento della pandemia tra diverse regioni\\
				\item Confrontare l'andamento della pandemia tra diversi periodi temporali\\
			\end{itemize}
		\end{frame}


	\section{Analisi risorse esistenti}
		\subsection{Linee guida}

			\begin{frame}
	 			\frametitle{Linee guida}
				Abbiamo individuato un totale di \textbf{42} linee guida che si adattano alla progettazione di una dashboard.
				\begin{block}{Fonti delle linee guida}
					\begin{itemize}[<+->]
						\item A. Dix et alii, HCI, Prentice Hall, 1998\\
						\item Jeff Johnson, Designing with the mind in mind, Morgan Kaufmann, 2010\\
						\item D. Egan, “Individual differences in HCI”, in M. Helander (ed.), Handbook of HCI, North-Holland, 1988\\
						\item A. Vaisman, E. Zimanyi. Data Warehouse System: Design and Implementation. Springer, 2016\\
						\item The guidelines of UserFocus.co.uk (commercial, UK, 2014)\\
						\item The 10 heuristics of Nielsen and Molich (1994)\\
						\item The 20 heuristics of Weinshenk and Barker (2000)\\
					\end{itemize}
				\end{block}
			\end{frame}
		
			\begin{frame}
				\frametitle{Valutazione delle risorse esistenti}
				Abbiamo analizzato la dashboard corrente individuandone pregi e difetti. \newline \newline
				Abbiamo individuato \textbf{23/42} linee guida non rispettate e molti altri problemi esaminando le varie componenti della dashboard attuale. 
			\end{frame}

			\begin{frame}
				IMMAGINE DASHBOARD CON EVIDENZIATI GLI ERRORI (1/2)
			\end{frame}

			\begin{frame}
				IMMAGINE DASHBOARD CON EVIDENZIATI GLI ERRORI (2/2)
			\end{frame}


		\subsection{Testing con gli utenti}

			\begin{frame}
				\frametitle{Testing con gli utenti}
				Abbiamo chiesto ad alcuni possibili utenti della dashboard di svolgere i task sulla dashboard attuale per capire quali migliorie apportare. \newline \newline
				Abbiamo adottato l'approccio Guerrilla Usability Testing per registrare gli errori compiuti dai tester così da poter definire una curva delle urgenze.
			\end{frame}

			\begin{frame}
				\frametitle{Curva delle urgenze}
				\begin{columns}[t]
					\begin{column}{5cm}
						Abbiamo valutato i dati raccolti calcolando la frequenza degli errori riuscendo a tracciare una curva di urgenza in termini di frequenza e impatto dei problemi emersi.
					\end{column}
					\begin{column}{5cm}
						IMMAGINE CURVA DELLE URGENZE
					\end{column}
				\end{columns}
			\end{frame}
		 
			\begin{frame}
				\frametitle{Roadmap}
				\begin{itemize}[<+->]
					\item \textbf{1}: Impossibile trovare il dato ``Tamponi effettuati''\\
					\item \textbf{3}: Impossibile trovare il dato relativo alle disponibilità nelle strutture ospedaliere\\
					\item \textbf{6}: Presenza di etichette poco chiare\\
					\item \textbf{5}: Non tutti i dati sono disponibili nella dashboard
					\item \textbf{4}: \`E lasciato all'utente il compito di computare il tasso di letalità\\
					\item \textbf{7}: \`E lasciato all'utente il compito di computare gli incrementi in un periodo di tempo\\
					\item \textbf{2}: Alcuni dati sono presenti nel PDF raggiungibile ``Schede riepilogo PDF''\\
				\end{itemize}
			\end{frame}

	\section{Studio di fattiblità}
		\subsection{Scenari}
		\begin{frame}
			\frametitle{Scenari}
			\begin{itemize}[<+->]
				\item Analisi dei dati quotidiani sull'andamento dell'epidemia Covid-19 in Italia\\
				\item Analisi dei dati quotidiani sull'andamento dell'epidemia Covid-19 della regione Emilia Romagna\\
				\item Analisi sull'andamento del tasso di letalità e dei numeri dell'epidemia nelle due settimane precedenti\\
				\item Analisi delle differenze dell'andamento nelle regioni Emilia Romagna, Veneto e Molise nelle precedenti due settimane\\
				\item Analisi delle differenze dell'andamento dell'epidemia in Italia nel mese corrente rispetto ai due precedenti\\
			\end{itemize}
		\end{frame}

		\subsection{Personas}
		\begin{frame}
			\frametitle{Personas}
		\end{frame}
		
	\section{Progettazione proposta}
		%\subsection{Design model}
		\begin{frame}
			\frametitle{Design model - CAO=S}
			\begin{enumerate}[<+->]
				\item Abbiamo individuato \textbf{8} concetti formati dai dati relativi alla pandemia liberamente disponibili e già usati in altre dashboard\\
				\item Utilizzando le persona e gli scenari abbiamo individuato gli attori e, per ognuno di essi abbiamo realizzato abbiamo realizzato un diagramma C\&A\\
				\item Abbiamo individuato \textbf{7} operazioni che gli attori possono svolgere\\
				\item Abbiamo individuato le strutture definendo quale attore potesse svolgere quale operazione\\
				\item Abbiamo individuato le strutture dati che dovrebbero essere utilizzate per gestire i diversi tipi di dato\\
			\end{enumerate}
		\end{frame}

		%\subsection{Architettura delle informazioni}
		\begin{frame}
			\frametitle{Architettura delle informazioni}
			\begin{itemize}[<+->]
				\item Approccio \textbf{top-down}\\
				\item Caratteristiche individuate:\\
					\begin{itemize}[<+->]
						\item Strutturazione: elementi interattivi\\
						\item Classificazioni: dati aggregati, dati assoluti, dati relativi, serie temporali\\
						\item Organizzazione: utilizzo di componenti grafiche distinte\\
						\item Ricercabilità: non presenti in quanto è mostrato l'intero contenuto informativo\\
						\item Gestione: configurabilità del layout per visioni personalizzate\\
					\end{itemize}
				\item Struttura di tipo \textbf{table of content}\\
			\end{itemize}
		\end{frame}

		%\subsection{Interaction design}
		\begin{frame}
			\frametitle{Interaction design}		
			\begin{itemize}[<+->]
				\item \textbf{Ergonomia}:
				\begin{itemize}[<+->]
					\item Organizzazione controlli e display: raggruppamento per frequenza\\
					\item Condizione ambiente fisico: non impattabile in quanto applicazione web-based\\
					\item Uso dei colori: evitato i colori come unico mezzo per differenziare le componenti\\
				\end{itemize} 
				\item \textbf{Progettazione della conversazione}:
					\begin{enumerate}[<+->]
						\item Menù e navigazione
						\item Manipolazione diretta
					\end{enumerate}
				\item \textbf{Progettazione delle schermate}
				\item \textbf{Contesto sociale e dell'organizzazione}
			\end{itemize}
		\end{frame}

		\subsection{Blueprint}
		\begin{frame}
			\frametitle{Blueprint}
			La realizzazione dei bluprint ha avuto necessità di tre iterazione che ci hanno portato a produrre un blueprint per comprendere quali contenuto devono esserci all'interno di ogni pagina e un blueprint per i programmatori che indichi quali componenti sono presenti all'interno di ogni pagina.
		\end{frame}

		\begin{frame}
			IMMAGINE BLUEPRINT v1
		\end{frame}

		\begin{frame}
			IMMAGINI BLUEPRINT v2
		\end{frame}

		\begin{frame}
			IMMAGINi BLUEPRINT v3 finale
		\end{frame}


		\subsection{Wireframe}
		\begin{frame}
			\frametitle{Wireframe}
			Utilizzando Balsamiq abbiamo realizzato dei wireframe delle diverse pagine 
		\end{frame}

		\begin{frame}
			IMMAGINE WIREFRAME
		\end{frame}

		\begin{frame}
			IMMAGINE WIREFRAME
		\end{frame}

		\begin{frame}
			IMMAGINE WIREFRAME
		\end{frame}


	\section{Valutazione}
		\subsection{Ispezione}
		\begin{frame}
			\frametitle{Cognitive walkthrough}
			Abbiamo svolto i task con la tecnica del cognitive walkthrough individuando alcune problematiche presenti nei wireframe che abbiamo corretto in successive iterazioni:
			\begin{itemize}[<+->]
				\item Diminuito il carico cognitivo aggiungendo etichette o semplifico la comprensibilità di quanto scritto\\
				\item Aggiunti dati necessari al completamento di alcuni task\\
				\item Modificato funzionamento relativo alla personalizzazione e alla condivisione dell'interfaccia\\
			\end{itemize}
		\end{frame}

		\begin{frame}
			\frametitle{Informal Action Analysis}
			Abbiamo svolto i task misurando, in modo non formale, quanto tempo ci si dovrebbe impiegare per svolgere le singolo azioni atomiche. \newline \newline
			\begin{block}{Risultati informal action analysis}
			Le tempistiche da noi misurate sono state in linea con quanto avevamo immaginato.
		\end{block}
		\end{frame}

		\begin{frame}
			\frametitle{Heuristic Analysis}
			Abbiamo confrontato le linee guida individuate con la nostra ri-progettazione. \newline \newline
			\begin{block}{Risultati heuristic analysis}
				Abbiamo individuato nove violazione che abbiamo prontamente corretto in una successiva iterazione dei wireframe.
		\end{block}
		\end{frame}

		\subsection{Testing}
		\begin{frame}
			\frametitle{Testing con gli utenti}
			\begin{itemize}[<+->]
				\item Abbiamo adottato un approccio di tipo Guerrilla\\ 
				\item Abbiamo definito un protocollo\\
				\item Abbiamo definito quali task sottoporre agli utenti:
					\begin{itemize}[<+->]
						\item Calcolare il tasso di positività relativo al 17 novembre 2020\\
						\item Valutare l'occupazione delle strutture sanitarie al 17 novembre 2020\\
						\item Calcolare il tasso di letalità medio nel mese di aprile 2020\\
					\end{itemize}
				\item Abbiamo provato a eseguire i task individuati nel test pilota\\
				\item Abbiamo chiesto a tre giornalisti di eseguire un task adottando l'approccio \textbf{thinking aloud}\\
			\end{itemize}
		\end{frame}

		\begin{frame}
			\frametitle{Testing con gli utenti - risultati}
			Abbiamo individuato \textbf{6} problematiche che abbiamo prontamente corretto in una successiva iterazione dei wireframe:
			\begin{enumerate}[<+->]
				\item Non si riesce a individuare il valore relativo ai ricoverati con sintomi\\
				\item Il ``secondo valore'' presente in ogni box numerico della schermata ``Panoramica'' non viene individuato immediatamente\\
				\item Richiesta di poter distinguere tra tamponi molecolari e antigenici\\
				\item Alcune metriche scelte di default nella tabella in ``Panoramica'' si sono rilevate poco interessanti\\
				\item Il termine ``metriche'' si è rilevato poco chiaro\\
				\item Difficoltà nel comprendere il funzionamento dei pulsanti per l'import e l'export dell'organizzazione dell'interfaccia\\
			\end{enumerate}
		\end{frame}

	\section{Conclusioni}
		\begin{frame}
			\frametitle{Conclusioni}
			\begin{block}{Obiettivo}
				Fornire il giusto contesto alle analisi dei giornalisti così da permetter loro di maturare comprensioni corrette e profonde delle metriche epidemiologiche
			\end{block}
			\begin{itemize}[<+->]
				\item Ricerca etnografica $\rightarrow$ esigenze e testimonianze dei giornalisti\\
				\item Analisi risorse esistenti $\rightarrow$ criticità e loro importanza\\
				\item Studio di fattibilità $\rightarrow$ contesto d'uso, task e persona\\
				\item Riprogettazione proposta $\rightarrow$ modello CAO=S, architettura delle informazioni, interazioni, blueprint e wireframe\\
				\item Inspection e testing $\rightarrow$ validazione di quanto ri-progettato mediante test interni al gruppo e interviste ai giornalisti\\
			\end{itemize}
		\end{frame}
\end{document}
